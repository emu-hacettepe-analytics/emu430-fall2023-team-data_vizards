% Options for packages loaded elsewhere
\PassOptionsToPackage{unicode}{hyperref}
\PassOptionsToPackage{hyphens}{url}
\PassOptionsToPackage{dvipsnames,svgnames,x11names}{xcolor}
%
\documentclass[
  11pt,
  a4paper,
  DIV=11,
  numbers=noendperiod]{scrartcl}

\usepackage{amsmath,amssymb}
\usepackage{iftex}
\ifPDFTeX
  \usepackage[T1]{fontenc}
  \usepackage[utf8]{inputenc}
  \usepackage{textcomp} % provide euro and other symbols
\else % if luatex or xetex
  \usepackage{unicode-math}
  \defaultfontfeatures{Scale=MatchLowercase}
  \defaultfontfeatures[\rmfamily]{Ligatures=TeX,Scale=1}
\fi
\usepackage{lmodern}
\ifPDFTeX\else  
    % xetex/luatex font selection
\fi
% Use upquote if available, for straight quotes in verbatim environments
\IfFileExists{upquote.sty}{\usepackage{upquote}}{}
\IfFileExists{microtype.sty}{% use microtype if available
  \usepackage[]{microtype}
  \UseMicrotypeSet[protrusion]{basicmath} % disable protrusion for tt fonts
}{}
\makeatletter
\@ifundefined{KOMAClassName}{% if non-KOMA class
  \IfFileExists{parskip.sty}{%
    \usepackage{parskip}
  }{% else
    \setlength{\parindent}{0pt}
    \setlength{\parskip}{6pt plus 2pt minus 1pt}}
}{% if KOMA class
  \KOMAoptions{parskip=half}}
\makeatother
\usepackage{xcolor}
\usepackage[lmargin=2cm,rmargin=2cm,tmargin=2cm,bmargin=2cm]{geometry}
\setlength{\emergencystretch}{3em} % prevent overfull lines
\setcounter{secnumdepth}{-\maxdimen} % remove section numbering
% Make \paragraph and \subparagraph free-standing
\ifx\paragraph\undefined\else
  \let\oldparagraph\paragraph
  \renewcommand{\paragraph}[1]{\oldparagraph{#1}\mbox{}}
\fi
\ifx\subparagraph\undefined\else
  \let\oldsubparagraph\subparagraph
  \renewcommand{\subparagraph}[1]{\oldsubparagraph{#1}\mbox{}}
\fi


\providecommand{\tightlist}{%
  \setlength{\itemsep}{0pt}\setlength{\parskip}{0pt}}\usepackage{longtable,booktabs,array}
\usepackage{calc} % for calculating minipage widths
% Correct order of tables after \paragraph or \subparagraph
\usepackage{etoolbox}
\makeatletter
\patchcmd\longtable{\par}{\if@noskipsec\mbox{}\fi\par}{}{}
\makeatother
% Allow footnotes in longtable head/foot
\IfFileExists{footnotehyper.sty}{\usepackage{footnotehyper}}{\usepackage{footnote}}
\makesavenoteenv{longtable}
\usepackage{graphicx}
\makeatletter
\def\maxwidth{\ifdim\Gin@nat@width>\linewidth\linewidth\else\Gin@nat@width\fi}
\def\maxheight{\ifdim\Gin@nat@height>\textheight\textheight\else\Gin@nat@height\fi}
\makeatother
% Scale images if necessary, so that they will not overflow the page
% margins by default, and it is still possible to overwrite the defaults
% using explicit options in \includegraphics[width, height, ...]{}
\setkeys{Gin}{width=\maxwidth,height=\maxheight,keepaspectratio}
% Set default figure placement to htbp
\makeatletter
\def\fps@figure{htbp}
\makeatother

\KOMAoption{captions}{tableheading}
\makeatletter
\makeatother
\makeatletter
\makeatother
\makeatletter
\@ifpackageloaded{caption}{}{\usepackage{caption}}
\AtBeginDocument{%
\ifdefined\contentsname
  \renewcommand*\contentsname{Table of contents}
\else
  \newcommand\contentsname{Table of contents}
\fi
\ifdefined\listfigurename
  \renewcommand*\listfigurename{List of Figures}
\else
  \newcommand\listfigurename{List of Figures}
\fi
\ifdefined\listtablename
  \renewcommand*\listtablename{List of Tables}
\else
  \newcommand\listtablename{List of Tables}
\fi
\ifdefined\figurename
  \renewcommand*\figurename{Figure}
\else
  \newcommand\figurename{Figure}
\fi
\ifdefined\tablename
  \renewcommand*\tablename{Table}
\else
  \newcommand\tablename{Table}
\fi
}
\@ifpackageloaded{float}{}{\usepackage{float}}
\floatstyle{ruled}
\@ifundefined{c@chapter}{\newfloat{codelisting}{h}{lop}}{\newfloat{codelisting}{h}{lop}[chapter]}
\floatname{codelisting}{Listing}
\newcommand*\listoflistings{\listof{codelisting}{List of Listings}}
\makeatother
\makeatletter
\@ifpackageloaded{caption}{}{\usepackage{caption}}
\@ifpackageloaded{subcaption}{}{\usepackage{subcaption}}
\makeatother
\makeatletter
\@ifpackageloaded{tcolorbox}{}{\usepackage[skins,breakable]{tcolorbox}}
\makeatother
\makeatletter
\@ifundefined{shadecolor}{\definecolor{shadecolor}{rgb}{.97, .97, .97}}
\makeatother
\makeatletter
\makeatother
\makeatletter
\makeatother
\ifLuaTeX
  \usepackage{selnolig}  % disable illegal ligatures
\fi
\IfFileExists{bookmark.sty}{\usepackage{bookmark}}{\usepackage{hyperref}}
\IfFileExists{xurl.sty}{\usepackage{xurl}}{} % add URL line breaks if available
\urlstyle{same} % disable monospaced font for URLs
\hypersetup{
  pdftitle={TEAM DATA VIZARDS},
  colorlinks=true,
  linkcolor={blue},
  filecolor={Maroon},
  citecolor={Blue},
  urlcolor={Blue},
  pdfcreator={LaTeX via pandoc}}

\title{TEAM DATA VIZARDS}
\author{}
\date{}

\begin{document}
\maketitle
\ifdefined\Shaded\renewenvironment{Shaded}{\begin{tcolorbox}[frame hidden, breakable, borderline west={3pt}{0pt}{shadecolor}, sharp corners, boxrule=0pt, interior hidden, enhanced]}{\end{tcolorbox}}\fi

This is our project webpage.

Please stay tuned to follow our project activities.

\hypertarget{team-members}{%
\section{TEAM MEMBERS}\label{team-members}}

\begin{enumerate}
\def\labelenumi{\arabic{enumi}.}
\item
  \href{https://emu-hacettepe-analytics.github.io/emu430-fall2023-sadiye-1/}{Şadiye
  Öztürk}
\item
  \href{https://emu-hacettepe-analytics.github.io/emu430-fall2023-melikekutlusan/}{Melike
  Nur Kutlusan}
\item
  \href{https://emu-hacettepe-analytics.github.io/emu430-fall2023-zeyneptnc/}{Zeynep
  Tuncaboylu}
\item
  \href{https://emu-hacettepe-analytics.github.io/emu430-fall2023-beyzanurnas/}{Beyzanur
  Nas}
\item
  \href{https://github.com/emu-hacettepe-analytics/emu430-fall2023-dygekn}{Duygu
  Eken}
\item
  \href{https://github.com/emu-hacettepe-analytics/emu430-fall2023-Meleker.git}{Melek
  Er}
\end{enumerate}

\hypertarget{project-topic}{%
\section{PROJECT TOPIC}\label{project-topic}}

We will carry out our project by considering 2021 internal migration
data across Turkey. Our main focus is In Migration of internal cities.
In this regard, migration trends between different regions such as
Eastern Anatolia, Southeastern Anatolia, Aegean, Marmara will be
determined and migration relations between big cities and other cities
will be analyzed in detail. The relationship between population density
and migration will be determined by examining migration rates,
especially in cities with large populations. Migration trends will be
evaluated according to age groups, gender, educational status and
reasons for migration. (retirement, appointment, etc.) Also, the
economic development level of the economic region will be interpreted
according to the labor force, household disposable income, number of
enterprises, number of housing sales, etc. This comprehensive analysis
will guide our project in understanding Turkey's migration situation.

\hypertarget{data-set}{%
\section{DATA SET}\label{data-set}}

\href{https://github.com/emu-hacettepe-analytics/emu430-fall2023-team-data_vizards/raw/master/Dataframe/migration_data.RData}{Migration
Data}

\href{https://github.com/emu-hacettepe-analytics/emu430-fall2023-team-data_vizards/raw/master/Dataframe/population_data.RData}{Population}

\href{https://github.com/emu-hacettepe-analytics/emu430-fall2023-team-data_vizards/raw/master/Dataframe/region26_data.RData}{Data
of 26 Region}

\hypertarget{key-takeaways}{%
\section{KEY TAKEAWAYS}\label{key-takeaways}}

The aim of this project is to analyze internal migration data in Turkey
and investigate the factors such as migration reasons, age, education,
and other variables that influence migration. We conducted our research
using the internal migration and population datasets available on the
Turkish Statistical Institute (TÜİK) website. The conclusions drawn from
this project are outlined in the following points:

\begin{enumerate}
\def\labelenumi{\arabic{enumi}.}
\tightlist
\item
  We had assumed that the highest migration due to retirement would be
  to the Central Anatolia Region or the Aegean Region. To validate this
  assumption, we created a migration value and cause graph. When looking
  at this graph, we observed that the highest migration is to the
  Marmara Region. To obtain more accurate results, we created a
  migration rate and cause graph. In this graph, we see that the highest
  migration is to the Black Sea and Aegean regions. So, we realized that
  part of our assumption is correct. Understanding the importance of
  creating a Rate graph, we concluded that we should make decisions
  based on this graph.
\item
  The age range of the people who migrate the most is seen in the graph
  as 20-24. The age group of the people who migrated the second most is
  seen in the graph as 15-19. In the graph of reasons for migration, the
  most common reason for migration was observed to be education. We can
  reconcile this incident with this situation.
\item
  The visual representation of migration numbers across different
  educational backgrounds highlights a compelling inverse relationship
  between education and migration. Doctoral and primary school-educated
  individuals show lower migration rates compared to high school and
  university-educated individuals. Individuals with higher education
  levels tend to migrate less. The scatter plot analysis reveals a lack
  of a straightforward correlation between the average education level
  and fertility rates across regions in the dataset. While some general
  trends emerge---like higher education levels in certain regions
  coinciding with lower fertility rates---outliers such as Central
  Anatolia and the Mediterranean Region defy these patterns.
\end{enumerate}



\end{document}
